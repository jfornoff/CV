%% start of file `template.tex'.
%% Copyright 2006-2015 Xavier Danaux (xdanaux@gmail.com).
%
% This work may be distributed and/or modified under the
% conditions of the LaTeX Project Public License version 1.3c,
% available at http://www.latex-project.org/lppl/.


\documentclass[11pt,a4paper,sans]{moderncv}        % possible options include font size ('10pt', '11pt' and '12pt'), paper size ('a4paper', 'letterpaper', 'a5paper', 'legalpaper', 'executivepaper' and 'landscape') and font family ('sans' and 'roman')
\usepackage[ngerman]{babel}
% moderncv themes
\moderncvstyle{banking}                             % style options are 'casual' (default), 'classic', 'banking', 'oldstyle' and 'fancy'
\moderncvcolor{black}                               % color options 'black', 'blue' (default), 'burgundy', 'green', 'grey', 'orange', 'purple' and 'red'
%\renewcommand{\familydefault}{\sfdefault}         % to set the default font; use '\sfdefault' for the default sans serif font, '\rmdefault' for the default roman one, or any tex font name
%\nopagenumbers{}                                  % uncomment to suppress automatic page numbering for CVs longer than one page

% character encoding
\usepackage[utf8]{inputenc}
\usepackage[scale=0.85]{geometry}
%\setlength{\hintscolumnwidth}{3cm}                % if you want to change the width of the column with the dates
%\setlength{\makecvtitlenamewidth}{10cm}           % for the 'classic' style, if you want to force the width allocated to your name and avoid line breaks. be careful though, the length is normally calculated to avoid any overlap with your personal info; use this at your own typographical risks...

% personal data
\name{Jan}{Fornoff}
\address{Amselweg 66}{64295 Darmstadt}{}% optional, remove / comment the line if not wanted; the "postcode city" and "country" arguments can be omitted or provided empty
\phone[mobile]{\texttt{+}49 163 4684860}                   % optional, remove / comment the line if not wanted; the optional "type" of the phone can be "mobile" (default), "fixed" or "fax"
%\phone[fixed]{06152 8553965}
%\phone[fax]{+3~(456)~789~012}
\email{janfornoff@gmail.com}
\title{Curriculum Vitae}                               % optional, remove / comment the line if not wanted
                           % optional, remove / comment the line if not wanted
\homepage{blog.janfornoff.com}                         % optional, remove / comment the line if not wanted
\social[linkedin]{jan-fornoff}                        % optional, remove / comment the line if not wanted
%\social[twitter]{jdoe}                             % optional, remove / comment the line if not wanted
\social[github]{jfornoff}                              % optional, remove / comment the line if not wanted
%\extrainfo{additional information}                 % optional, remove / comment the line if not wanted
%\photo[64pt][0.4pt]{picture}                       % optional, remove / comment the line if not wanted; '64pt' is the height the picture must be resized to, 0.4pt is the thickness of the frame around it (put it to 0pt for no frame) and 'picture' is the name of the picture file
%\quote{Some quote}                                 % optional, remove / comment the line if not wanted

% bibliography adjustements (only useful if you make citations in your resume, or print a list of publications using BibTeX)
%   to show numerical labels in the bibliography (default is to show no labels)
\makeatletter\renewcommand*{\bibliographyitemlabel}{\@biblabel{\arabic{enumiv}}}\makeatother
%   to redefine the bibliography heading string ("Publications")
%\renewcommand{\refname}{Articles}

% bibliography with mutiple entries
%\usepackage{multibib}
%\newcites{book,misc}{{Books},{Others}}
%----------------------------------------------------------------------------------
%            content
%----------------------------------------------------------------------------------
\begin{document}
%\begin{CJK*}{UTF8}{gbsn}                          % to typeset your resume in Chinese using CJK
%-----       resume       ---------------------------------------------------------
\makecvtitle
\pagestyle{empty}

\section{Education}
\cventry{2016--2018}{TU Darmstadt}{MSc Business Information Systems}{}{}{
  Master Thesis: ``Evaluating Quality of Experience in Multipath TCP Applications on iOS Mobile Devices'' (Graded 1.0)
  \begin{itemize}
    \item Development of iOS application capable of running web applications over MPTCP
    \item Design of network testbed allowing per-path network emulation and using real application from the Internet
    \item Conducting a user study while collecting extensive metrics and statistics for analysis
  \end{itemize}
}{}{}
\cventry{2011--2016}{TU Darmstadt}{BSc Business Information Systems}{}{Average grade \textit{2.27}}{Bachelor Thesis: ``Optimization and Evaluation of MPEG DASH Video Streaming by large-scale Network Simulation'' \\ (Graded 1.0 \& nominated for the Datenlotsen award for outstanding theses)}{}{}
\cventry{2007--2010}{Berufliches Gymnasium Groß-Gerau}{High school graduation (Abitur)}{}{Average grade \textit{1.5}}{Focused on Information Systems / Electrical Engineering, Intensive course: Maths}  % arguments 3 to 6 can be left empty

\section{Scientific publications}
\cventry{}{Denny Stohr, Alexander Fr"ommgen, Jan Fornoff et al.}{\emph{QoE Analysis of DASH Cross-Layer Dependencies by Extensive Network Emulation }}{}{ACM SIGCOMM 2016 (Workshop)}{}
\cventry{}{Alexander Fr"ommgen, Denny Stohr, Jan Fornoff et al.}{\emph{Capture and Replay: Reproducible Network Experiments in Mininet}}{}{ACM SIGCOMM 2016 (Poster)}{}

\section{Experience}
\cventry{Sep. 2017--present}{Working student Backend Development (Remote)}{Grandcentrix GmbH}{Cologne}{}{
  \smallskip
  \begin{itemize}
    \item Elixir application development (OTP, Phoenix framework)
    \item Extensive use of CQRS / Event Sourcing paradigms
    \item Taking part in software architectural discussions
    \item Load testing, profiling and fixing performance bottlenecks
    \item Diagnosing and troubleshooting failures on live Kubernetes deployment
    \item Participating in internal expert communities to advance engineering practices
  \end{itemize}
}
\cventry{Oct. 2011--Sep. 2017}{Working student}{Software AG}{Eberstadt}{}{Development of internal web applications \newline{}%
\begin{itemize}
\item Development of web applications for management of contracts
\item Automation of Non-Disclosure Agreement generation
\item Development of web application for creating dynamically configurable questionnaires
\item Integration with internal data and authentication APIs
\item Conception of a Docker-based deployment and application operation platform
\item Linux system administration, deployment automation \& monitoring
\end{itemize}}
\cventry{Aug. 2011--Oct. 2011}{Obligatory internship for course of study}{Software AG}{Eberstadt}{}{}

\section{Projects}
\subsection{\textbf{University projects}}
\smallskip
\cventry{}{}{Course project ``IT project management''}{Invensity GmbH
  (Wiesbaden)}{}{Development of single-page application to collect customer feedback (AngularJS 1.x - Frontend, Laravel - Backend)}

\medskip

\cventry{}{}{Course project ``Creating Value through Data Science''}{Accenture Digital}{}{Analytics challenge with the goal of getting insights into the profession of Data Scientists. \newline{}
\begin{itemize}
  \item Business case definition
  \item Data preparation for analysis
  \item Statistical data analysis and modeling
  \item Development of prediction model
  \item Data visualization
  \item Presentation of results in front of experts from science and practice
\end{itemize}
}

\subsection{\textbf{Personal side-projects}}
\smallskip
\cventry{}{}{Chatbot for asynchronous stand-up meetings}{\href{https://docs.google.com/presentation/d/1EwYcrZ7avlwdU9C6yntaoqN5ANCbapg_Gp7QJqb5ACo/pub?slide=id.p}{Meetup-Slides}}{}{Development of Chatbot based on ``Slack'', who conducts asynchronous stand-up meetings at user-defined times by direct message and posts the formatted results into a status chat room. (Elixir / OTP)}

\medskip

\cventry{}{}{Chefkoch-Shopping}{}{}{Experimental project to grasp Node.JS principles. \newline{}
Captures a Chefkoch (cooking recipe site) recipe, crawls it for ingredients and allows adding them to a shopping list.
\begin{itemize}
  \item OAuth \& REST API integration with Wunderlist
  \item Crawling of recipe site for ingredients
  \item Docker-based deployment
  \item Express.JS on top of Node.JS
\end{itemize}
}

\section{Languages}
\cvitemwithcomment{English}{fluent (spoken and written)}{}
\cvitemwithcomment{German}{Mother tongue level}{}

\section{Technical skills}
\subsection{\textbf{Programming languages}}
\smallskip
\cvdoubleitem{Advanced level}{
\begin{itemize}
\item Elixir
\item Elm
\item Ruby
\item HTML / JavaScript
\end{itemize}}
{Basics}{
\begin{itemize}
\item Python
\item Go
\item Java
\item Scala
\item C(++)
\item Assembler
\item Rust
\item Haskell
\item R
\end{itemize}}
\subsection{\textbf{Technologies}}
\smallskip
\cvdoubleitem{Tools}
{
\begin{itemize}
\item Git
\item Docker
\item Kubernetes
\item Rancher
\item Gitlab CI
\end{itemize}
}
{Frameworks}
{
\begin{itemize}
\item Phoenix (Elixir)
\item Ruby on Rails
\item node.js / Express
\item ReactJS
\item Laravel
\end{itemize}
}

\medskip

\cvdoubleitem{Storage}
{
\begin{itemize}
\item PostgreSQL
\item MySQL
\item Redis
\item RabbitMQ
\end{itemize}
}
{Protocols}
{
  \begin{itemize}
    \item HTTP-based adaptive bitrate video streaming (MPEG-DASH / HLS)
    \item Network protocols (HTTP, TCP/IP, Multipath TCP)
  \end{itemize}
}

\section{Technical areas of interest}
\cvlistitem{Web Technologies \& Future Internet}
\cvlistitem{System performance optimization (across abstraction boundaries)}
\cvlistitem{Agile \& Test-Driven Development}
\cvlistitem{Systems engineering \& SRE}
\cvlistitem{Big Data, Data Mining \& Statistical Analysis}

\section{Hobbys}
\cvlistitem{
Sports
\begin{itemize}
  \item Handball
  \item Strength training
\end{itemize}
}
\cvlistitem{Cooking}
\cvlistitem{Traveling and hiking}
\end{document}
